\documentclass[11pt, letterpaper]{amsart}
\usepackage{mathtools, stmaryrd, wasysym}
\usepackage{pdfsync,url, mathrsfs}
\usepackage{enumerate}
\usepackage{amsmath}
\usepackage{amssymb}
\usepackage{amsthm}
  \theoremstyle{plain}
  \newtheorem{Theorem}{Theorem}
  \newtheorem{Lemma}[Theorem]{Lemma}
  \newtheorem{Corollary}[Theorem]{Corollary}
  \newtheorem{Proposition}[Theorem]{Proposition}
  \theoremstyle{definition}
  \newtheorem{defn}[Theorem]{Definition}
  \newtheorem{Example}[Theorem]{Example}
  \theoremstyle{remark}
  \newtheorem{Remark}[Theorem]{Remark}
\usepackage[T1]{fontenc}
\usepackage{palatino}
\linespread{1.05}
\renewcommand{\familydefault}{pplx}
\renewcommand{\rmdefault}{pplx}
\usepackage[euler-digits,small]{eulervm}
\usepackage[utf8]{inputenc}
\usepackage[tracking]{microtype}
\usepackage{stackrel,braket}
\usepackage{tikz-cd}

\setlength{\oddsidemargin}{0cm}
\setlength{\evensidemargin}{0cm}
\setlength{\marginparwidth}{0in}
\setlength{\marginparsep}{0in}
\setlength{\marginparpush}{0in}
\setlength{\topmargin}{0in}
\setlength{\headheight}{0pt}
\setlength{\headsep}{20pt}
\setlength{\footskip}{.3in}
\setlength{\textheight}{9.2in}
\setlength{\textheight}{8.8in}
\setlength{\textwidth}{6.5in}
\setlength{\parskip}{4pt}

\renewcommand{\qedsymbol}{$\blacksquare$}
\newcommand{\done}{/\!\!\!/}
\makeatletter
\newcommand*{\coloneqq}{\mathrel{\rlap{%
           \raisebox{0.3ex}{$\m@th\cdot$}}%
           \raisebox{-0.3ex}{$\m@th\cdot$}}%
           =}
\makeatother

\DeclarePairedDelimiter{\abs}{\lvert}{\rvert}
\DeclarePairedDelimiter{\norm}{\lVert}{\rVert}
\DeclarePairedDelimiter{\inner}{\langle}{\rangle}
\DeclarePairedDelimiter{\agen}{\langle}{\rangle}


\DeclareMathOperator{\Conv}{Conv}
\DeclareMathOperator{\Int}{int}
\DeclareMathOperator{\Relint}{relint}
\DeclareMathOperator{\Ext}{Ext}
\DeclareMathOperator{\Simple}{Simple}
\DeclareMathOperator{\Poly}{Poly}
\DeclareMathOperator{\Rank}{Rank}
\DeclareMathOperator{\Real}{Real}
\DeclareMathOperator{\Aff}{Aff}
\DeclareMathOperator{\Vol}{vol}
\DeclareMathOperator{\tr}{tr}
\DeclareMathOperator{\Ric}{Ric}
\DeclareMathOperator{\Res}{Res}
\DeclareMathOperator{\trdeg}{tr\,deg}
\DeclareMathOperator{\Kdim}{Kdim}
\DeclareMathOperator{\GKdim}{GKdim}
\DeclareMathOperator{\Frac}{Frac}
\DeclareMathOperator{\Ann}{Ann}
\DeclareMathOperator{\End}{End}
\DeclareMathOperator{\Aut}{Aut}
\DeclareMathOperator{\Hom}{Hom}
\DeclareMathOperator{\sgn}{sign}
\DeclareMathOperator*{\esssup}{ess\,sup}
\DeclareMathOperator{\coker}{coker}
\DeclareMathOperator{\im}{im}
\DeclareMathOperator{\Span}{span}
\DeclareMathOperator{\Spec}{Spec}

% Categories
\newcommand{\op}{\mathrm{op}}
\newcommand{\CSh}{\mathbold{Sh}}
\newcommand{\CAb}{\mathbold{Ab}}
\newcommand{\CModR}{\mathbold{Mod_R}}
\newcommand{\CTop}{\mathbold{Top}}
\newcommand{\CGrp}{\mathbold{Grp}}
\newcommand{\CRing}{\mathbold{Ring}}
\newcommand{\CSet}{\mathbold{Set}}

% Standard Sets
\newcommand{\id}{\mathrm{id}}
\newcommand{\ZZ}{\mathbold{Z}}
\newcommand{\ZP}{\mathbold{Z}_+}
\newcommand{\ZNN}{\mathbold{Z}_{\geq 0}}
\newcommand{\NN}{\mathbold{N}}
\newcommand{\QQ}{\mathbold{Q}}
\newcommand{\RR}{\mathbold{R}}
\newcommand{\RP}{\mathbold{R}_+}
\newcommand{\RPN}{\mathbold{R}_{\geq 0}}
\newcommand{\CC}{\mathbold{C}}
\newcommand{\FF}{\mathbold{F}}
\newcommand{\PP}{\mathbold{P}}
\newcommand{\KK}{\mathbold{K}}
\renewcommand{\AA}{\mathbold{A}}

\renewcommand{\Im}{\mathrm{Im\,}}
\renewcommand{\Re}{\mathrm{Re\,}}
\newcommand{\tensor}[1][]{\mathchoice%
  {\stackrel[#1]{}{\otimes}}%
  {\otimes_{#1}}%
  {\otimes_{#1}}%
  {\otimes_{#1}}%
}
\newcommand{\wc}{{\mkern 2mu\cdot\mkern 2mu}}
\newcommand{\idealeq}{\trianglelefteq}
\newcommand{\ideal}{\triangleleft}

\newcommand{\didi}[2][]{\frac{\partial #1}{\partial #2}}
\newcommand{\dd}[2][]{\frac{d #1}{d #2}}

% Mathcal letters
\newcommand{\cA}{\mathcal{A}}
\newcommand{\cB}{\mathcal{B}}
\newcommand{\cC}{\mathcal{C}}
\newcommand{\cD}{\mathcal{D}}
\newcommand{\cE}{\mathcal{E}}
\newcommand{\cF}{\mathcal{F}}
\newcommand{\cG}{\mathcal{G}}
\newcommand{\cH}{\mathcal{H}}
\newcommand{\cI}{\mathcal{I}}
\newcommand{\cJ}{\mathcal{J}}
\newcommand{\cK}{\mathcal{K}}
\newcommand{\cL}{\mathcal{L}}
\newcommand{\cM}{\mathcal{M}}
\newcommand{\cN}{\mathcal{N}}
\newcommand{\cO}{\mathcal{O}}
\newcommand{\cP}{\mathcal{P}}
\newcommand{\cQ}{\mathcal{Q}}
\newcommand{\cR}{\mathcal{R}}
\newcommand{\cS}{\mathcal{S}}
\newcommand{\cT}{\mathcal{T}}
\newcommand{\cU}{\mathcal{U}}
\newcommand{\cV}{\mathcal{V}}
\newcommand{\cW}{\mathcal{W}}
\newcommand{\cX}{\mathcal{X}}
\newcommand{\cY}{\mathcal{Y}}
\newcommand{\cZ}{\mathcal{Z}}

\newcommand{\fg}{\mathfrak{g}}
\newcommand{\fh}{\mathfrak{h}}
\newcommand{\fp}{\mathfrak{p}}
\newcommand{\fm}{\mathfrak{m}}
\newcommand{\fn}{\mathfrak{n}}
\newcommand{\fX}{\mathfrak{X}}
\newcommand{\fY}{\mathfrak{Y}}
\newcommand{\fZ}{\mathfrak{Z}}

\newcommand{\bS}{\mathbf{S}}

% Header
\title{CO759 Project: All the Same, I Saw it First}
\author{Raymond Cheng, Sam Eisenstat}
\date{\today}
\begin{document}
\maketitle

\section{Scaffolding}
We describe some of the considerations that we went through in the process of
developing the algorithm that we did. In particular, we will describe a little
web application built for the purpose of trying to manually find combs in a graph.

\subsection{Staring at the Mona Lisa}
A lot of the initial ideas came from staring long and hard at the $x$-vector
visualization of the Mona Lisa. One particular region of the image attracted
our eyes: the amply exposed region above her bosom. There hid a rather red
cycle. This reminded us of the prototypical example of a violated comb. Indeed,
at the time, we thought that we have found a violated comb in the Mona Lisa
image and this belief motivated a lot of what followed. On later inspection, of
course, we realized that the supposed violating comb in the Mona Lisa was not
actually a comb: it had an even number of teeth.

This particular incident suggested that combs should be found by looking for
subgraphs made up with mainly fractional edges and then growing all possible
teeth from them. This, of course, is not a new idea: heuristics of this sort
have been developed by various authors. The method that we will describe, as
simple as it turned out to be, does not seem to be documented in the very few
articles that we read. We apologize beforehand if it turns out that we have
done nothing but reproduce preexisting work.

\subsection{Manually Exploring Combs}
After studying the $x$-vector visualization of the Mona Lisa, we felt it would
be useful to be able to manually construct combs in a visualization and test
whether or not they violated the comb inequality. We set off to build a little
Javascript application with this purpose. The source code along with some
(poorly written) documentation on its use is included with this document.

In our manual search for comb inequalities, we first tried to construct a
handle and then grow teeth out of this handle, as many authors have previously
done so.  Motivated both by the idea that the handle should consist of nodes
joined by fractional edges of weight close to $0.5$ and the need to simplify an
otherwise very complicated graph, we tried to look for handles after hiding
edges of weight $1$. This quickly led to the idea that, when looking for
handles, it may be useful to hide edges of high weight in general. Some time
much later, we also explored hiding edges of low weight too. These ideas,
all arising from manual experimentation, form the basis of our heuristic.

\section{Comb Separation Heuristic: Theory}
After the quick discussion of the motivation and intuition underpinning our
heuristic, we make precise why our choices and algorithm makes sense. We will
quickly recall what combs are and analyze the comb inequality to see what a
violating comb must consist of. This analysis will naturally lead to what the
teeth and handle should look like in a violating comb.

\subsection{Combs}
Let us quickly recall the setup. Let $G = (V,E)$ be a weighted graph. A
\textbf{comb} in $G$ consists of a \textbf{handle} $H \subsetneq V$ and $2k+1$,
$k \geq 1$, \textbf{teeth} $T_1,\ldots,T_{2k+1} \subset V$ such that $\abs{H}
\geq 3$, the $T_i$ are pairwise disjoint, $T_i \cap H \neq \varnothing$ and
$T_i \setminus H \neq \varnothing$. We shall also call a tooth $T_i$
\textbf{simple} if $\abs{T_i} = 2$.

The comb inequalities discovered by Chv\'atal and Gr\"otschel \& Padberg is
equivalent to the following: if $x$ is the characteristic vector of any tour in
$G$, then
\begin{equation}\label{ineq:comb}
  x(\delta(H)) + \sum^{2k+1}_{i = 1}x(\delta(T_i)) \geq 3(2k+1) + 1.
\end{equation}
Let us carefully examine~\eqref{ineq:comb} to see how we might find a violating comb.
Without loss, we may assume that the graph under consideration, although
fractional, satisfies the degree LP and all subtour inequalities. Moreover, by
going through the derivation of~\eqref{ineq:comb}, it also turns out that the if $x$
violates~\eqref{ineq:comb}, then the right hand side is smaller than the left hand side by
no more than $1$.

\subsection{Violating Combs}
To find a violating comb in $G$, we require a comb which makes the left hand
side of~\eqref{ineq:comb} as large as possible, while keeping the right side small. The
right hand side is determined completely by the number of teeth in the comb, so
whenever we construct a comb, we would like to have as many teeth as possible.

Suppose now that we have set $H \subset V$ which is to be the handle of some
comb.  The preceding discussion suggests that we should grow as many teeth out
of $H$ as possible. Let $v \in H$ be a vertex incident with edges
$e_1,\ldots,e_m$ such that the other end $u_i$ of $e_i$ is not in $H$. Consider
a tooth formed by $v$ along with some subset of the $u_i$, say $i =
1,\ldots,l$. This tooth will contribute
\begin{equation*}
  \sum^m_{i = 1}w(e_i) + \sum^l_{i = 1} (4 - 2w(e_i)) = 4l - \sum^l_{i = 1} w(e_i) + \sum^m_{i = l + 1}w(e_i) \geq 3l
\end{equation*}
on the left hand side of~\eqref{ineq:comb}, and a factor of $3$ on the right. The final
inequality holds with equality if and only if $w(e_i) = 1$ for $i = 1,\ldots,l$
and $l = m$. By the degree constraints, we also see that the equality conditions
hold only when $l = 1$.

\subsection{Best Teeth}
This analysis suggests a way to build a comb given a handle $H$. The best
possible teeth on $H$ are simple teeth in which the associated edge has weight
$1$. Suppose that $r$ edges $e_1,\ldots,e_r$ in $\delta(H)$ have edge weight
$1$.  The above analysis shows that in order to have any hope of finding a
violated comb inequality, we must add to our comb each simple tooth defined by
$e_i$. In the fortunate case that $n$ is odd and the $e_i$ are all the edges
incident with $H$, the resulting comb $C$ is a violating comb:
\begin{equation*}
  x(\delta(H)) + \sum^r_{i = 1}x(\delta(e_i)) = r + (4 - 2)r = 3r < 3r + 1.
\end{equation*}

\subsection{Lower Weight Teeth}
In general, either $r$ is even there are edges $e'_1,\ldots,e'_s$ that are of
weight strictly less than $1$. In the first case, nothing can be done and $H$
is not the handle defining a violating comb. In the second case, a judicious
choice of which $e_i'$ to include as a tooth in $C$ could still give a violated
comb. To get a feel for which teeth should be added, consider what~\eqref{ineq:comb} looks
like in this case so far. With the comb $C$ as above,
\begin{equation*}
  x(\delta(H)) + \sum^r_{i = 1}x(\delta(e_i)) = 3r + \sum^s_{i = 1}w(e_i') \geq 3r + 1.
\end{equation*}
Note that $\sum^s_{i = 1}w(e_i) \geq 1$ due to the degree constraints---in
fact, this sum is equal to the number $l$ of vertices $v_1,\ldots,v_l \in H$
incident with at least one of the edges $e_i'$. In fact, this tells us that we
must include one tooth for each vertex $v_i$. Consequently, in our
consideration, the number of vertices in $H$ which are incident to something
not in $H$ must be odd.

Consider the case in which we include exactly one edge incident with $v_i$ for
each $i$. By reindexing the edges, assume that the new teeth included are those
determined by $e_1',\ldots,e_l'$. The inequality~\eqref{ineq:comb} for this new comb $C'$
is then
\begin{equation*}
  3r + \sum^r_{i = 1} w(e_i') + \sum^l_{i = 1}(4 - 2w(e_j')) = 3(r+l) + \sum^r_{i = l+1}w(e_i') + \left(l - \sum^l_{i = 1}w(e_i')\right) \stackrel{?}{\geq} 3(r+l) + 1.
\end{equation*}
Then $C'$ is a violating comb if and only if $s + l$ is odd and
\begin{equation}\label{ineq:2}
  \sum^r_{i = l+1} w(e_i') + \left(l - \sum^l_{i = 1}w(e_i')\right) < 1.
\end{equation}
This inequality tells us that the edges $e_i'$ to be included should be chosen
with $w(e_i')$ maximal amongst those edges incident to the same vertex $v_j$.


\subsection{Ignoring High and Low Weight Edges}
In the above discussion, we assumed we were given a handle $H$ and then we
considered the teeth that must be taken to potentially construct a violated
comb.  But we now face the problem of finding the initial handle $H$. Our
simple heuristic to find $H$ will be guided by the properties of the edges
leaving $H$.

We saw that, ideally, the teeth of a comb are formed by edges of weight almost
$1$. Concretely, this means that $H$ should be an induced subgraph of $G$ such
that the weight of the edges joining $H$ with the vertices of $G$ should be
close to $1$. Such a subgraph can be found by considering the graph $G'$ whose
vertices are the same as those of $G$ and whose edges are those of $E$ with
weight smaller than some parameter $\beta$ close to $1$. The connected
components of $G'$ will then be candidates for handles.

But not all edges leaving a handle must be high weight edges.
Indeed~\eqref{ineq:2} says that handles may have low weight edges leaving it
too. Thus it would perhaps be more effective to choose another parameter
$\alpha$ close to $0$ and set $G'$ to be the subgraph of $G$ with the same
vertex set and edge set consisting of edges $e \in E$ satisfying $\alpha < w(e)
< \beta$. Denote by $G(\alpha,\beta) \coloneqq G'$ the subgraph formed in this
way. Connected components of $G(\alpha,\beta)$ are candidate handles.

\subsection{Oddness Condition}
Clearly, not all connected components of $G(\alpha,\beta)$ will be the handle
of a comb. For instance, connected components of size $1$ and $2$ cannot be a
handle. Also, we have seen that the handle of a violating comb must have have
an odd number of vertices which are incident to a edge leaving the handle.
Thus, amongst all connected components of $G(\alpha,\beta)$, those which can be
considered as potential handles are those components of size at least $3$ and
which have odd number of vertices on its border.

\section{Computational Aspects}
Having described the graph theoretic considerations, the heuristic algorithm we
implemented is then a translation of the above into code. Since the code is attached
with this document, we will content ourselves with a high level sketch of what is
going on.

\subsection{Algorithm}
Let the input graph be $G' = (V',E')$ with weight function $w'$.
\begin{enumerate}
  \item Contract weight $1$ paths in $G'$ to obtain a new graph $G = (V,E)$
    with weight function $w$;
  \item Choose parameters $0 < \alpha < \beta < 1$;
  \item Compute connected components of $G(\alpha,\beta)$;
  \item For each connected component $H$ of $G(\alpha,\beta)$, construct a comb-like
    object $C$ by taking one simple tooth for every border vertex of $H$;
  \item Check whether $C$ is a valid comb and, in that case, check whether or not
    $C$ is violating.
\end{enumerate}

\subsection{Contracting Weight $1$ Paths}
Let us elaborate on the first step a little more, as we did not discuss this
when considering the graph theory above. In particular, we explain why this
computation simplication step does not affect the validity of a violating comb
found with this algorithm.  By ``contracting weight $1$ paths'', we mean that a
path in $G'$ consisting of weight $1$ edges is to be replaced by a single
weight $1$ edge.

First, this contraction step does not affect the contributions made by teeth of
combs found above. In our heuristic, we look only for simple teeth, i.e. those
teeth $T$ determined by a single edge $e$. By the degree constraints, the
contributions of $x(\delta(T))$ are necessarily $4 - w(e)$; in particular, this
depends only on the weight of $e$. Since we are simply replacing paths of
weight $1$ edges with a single weight $1$ edge, no change occurs to
$x(\delta(T))$ for any simple tooth $T$.

Second, contraction does not eliminate potential handles. A comb must have at
least $3$ teeth, so the handle of comb cannot be strictly contained in a weight
$1$ path: otherwise, the degree constraints would imply the handle could only
have $2$ teeth.

\subsection{Choice of Parameters}
There are two parameters, $\alpha$ and $\beta$, which need to be chosen in the
algorithm. Indeed, different choices of $\alpha$ and $\beta$ will make a
significant difference in terms of whether or not this method finds any
violating combs.

% Decrementing
% Just lowering to each occurrence -- sort the weights and jump
\end{document}
