\documentclass[11pt, letterpaper]{amsart}
\usepackage{mathtools, stmaryrd, wasysym}
\usepackage{pdfsync,url, mathrsfs}
\usepackage{enumerate}
\usepackage{amsmath}
\usepackage{amssymb}
\usepackage{amsthm}
  \theoremstyle{plain}
  \newtheorem{Theorem}{Theorem}
  \newtheorem{Lemma}[Theorem]{Lemma}
  \newtheorem{Corollary}[Theorem]{Corollary}
  \newtheorem{Proposition}[Theorem]{Proposition}
  \theoremstyle{definition}
  \newtheorem{defn}[Theorem]{Definition}
  \newtheorem{Example}[Theorem]{Example}
  \theoremstyle{remark}
  \newtheorem{Remark}[Theorem]{Remark}
\usepackage[T1]{fontenc}
\usepackage{mathpazo}
\linespread{1.05} % following Ian Morrison
\renewcommand{\familydefault}{pplx}
\renewcommand{\rmdefault}{pplx}
\usepackage[euler-digits,small]{eulervm}
\usepackage[utf8]{inputenc}
\usepackage[tracking]{microtype}
\usepackage{stackrel,braket}
\usepackage{tikz-cd}

\setlength{\oddsidemargin}{0cm} \setlength{\evensidemargin}{0cm}
\setlength{\marginparwidth}{0in}
\setlength{\marginparsep}{0in}
\setlength{\marginparpush}{0in}
\setlength{\topmargin}{0in}
\setlength{\headheight}{0pt}
\setlength{\headsep}{20pt}
\setlength{\footskip}{.3in}
%\setlength{\textheight}{9.2in}
\setlength{\textheight}{8.8in}
\setlength{\textwidth}{6.5in}
\setlength{\parskip}{4pt}

\renewcommand{\qedsymbol}{$\blacksquare$}
\newcommand{\done}{/\!\!\!/}
\makeatletter
\newcommand*{\coloneqq}{\mathrel{\rlap{%
           \raisebox{0.3ex}{$\m@th\cdot$}}%
           \raisebox{-0.3ex}{$\m@th\cdot$}}%
           =}
\makeatother

\DeclarePairedDelimiter{\abs}{\lvert}{\rvert}
\DeclarePairedDelimiter{\norm}{\lVert}{\rVert}
\DeclarePairedDelimiter{\inner}{\langle}{\rangle}
\DeclarePairedDelimiter{\agen}{\langle}{\rangle}


\DeclareMathOperator{\Conv}{Conv}
\DeclareMathOperator{\Int}{int}
\DeclareMathOperator{\Relint}{relint}
\DeclareMathOperator{\Ext}{Ext}
\DeclareMathOperator{\Simple}{Simple}
\DeclareMathOperator{\Poly}{Poly}
\DeclareMathOperator{\Rank}{Rank}
\DeclareMathOperator{\Real}{Real}
\DeclareMathOperator{\Aff}{Aff}
\DeclareMathOperator{\Vol}{vol}
\DeclareMathOperator{\tr}{tr}
\DeclareMathOperator{\Ric}{Ric}
\DeclareMathOperator{\Res}{Res}
\DeclareMathOperator{\trdeg}{tr\,deg}
\DeclareMathOperator{\Kdim}{Kdim}
\DeclareMathOperator{\GKdim}{GKdim}
\DeclareMathOperator{\Frac}{Frac}
\DeclareMathOperator{\Ann}{Ann}
\DeclareMathOperator{\End}{End}
\DeclareMathOperator{\Aut}{Aut}
\DeclareMathOperator{\Hom}{Hom}
\DeclareMathOperator{\sgn}{sign}
\DeclareMathOperator*{\esssup}{ess\,sup}
\DeclareMathOperator{\coker}{coker}
\DeclareMathOperator{\im}{im}
\DeclareMathOperator{\Span}{span}
\DeclareMathOperator{\Spec}{Spec}

% Categories
\newcommand{\op}{\mathrm{op}}
\newcommand{\CSh}{\mathbold{Sh}}
\newcommand{\CAb}{\mathbold{Ab}}
\newcommand{\CModR}{\mathbold{Mod_R}}
\newcommand{\CTop}{\mathbold{Top}}
\newcommand{\CGrp}{\mathbold{Grp}}
\newcommand{\CRing}{\mathbold{Ring}}
\newcommand{\CSet}{\mathbold{Set}}

% Standard Sets
\newcommand{\id}{\mathrm{id}}
\newcommand{\ZZ}{\mathbold{Z}}
\newcommand{\ZP}{\mathbold{Z}_+}
\newcommand{\ZNN}{\mathbold{Z}_{\geq 0}}
\newcommand{\NN}{\mathbold{N}}
\newcommand{\QQ}{\mathbold{Q}}
\newcommand{\RR}{\mathbold{R}}
\newcommand{\RP}{\mathbold{R}_+}
\newcommand{\RPN}{\mathbold{R}_{\geq 0}}
\newcommand{\CC}{\mathbold{C}}
\newcommand{\FF}{\mathbold{F}}
\newcommand{\PP}{\mathbold{P}}
\newcommand{\KK}{\mathbold{K}}
\renewcommand{\AA}{\mathbold{A}}

\renewcommand{\Im}{\mathrm{Im\,}}
\renewcommand{\Re}{\mathrm{Re\,}}
\newcommand{\tensor}[1][]{\mathchoice%
  {\stackrel[#1]{}{\otimes}}%
  {\otimes_{#1}}%
  {\otimes_{#1}}%
  {\otimes_{#1}}%
}
\newcommand{\wc}{{\mkern 2mu\cdot\mkern 2mu}}
\newcommand{\idealeq}{\trianglelefteq}
\newcommand{\ideal}{\triangleleft}

\newcommand{\didi}[2][]{\frac{\partial #1}{\partial #2}}
\newcommand{\dd}[2][]{\frac{d #1}{d #2}}

% Mathcal letters
\newcommand{\cA}{\mathcal{A}}
\newcommand{\cB}{\mathcal{B}}
\newcommand{\cC}{\mathcal{C}}
\newcommand{\cD}{\mathcal{D}}
\newcommand{\cE}{\mathcal{E}}
\newcommand{\cF}{\mathcal{F}}
\newcommand{\cG}{\mathcal{G}}
\newcommand{\cH}{\mathcal{H}}
\newcommand{\cI}{\mathcal{I}}
\newcommand{\cJ}{\mathcal{J}}
\newcommand{\cK}{\mathcal{K}}
\newcommand{\cL}{\mathcal{L}}
\newcommand{\cM}{\mathcal{M}}
\newcommand{\cN}{\mathcal{N}}
\newcommand{\cO}{\mathcal{O}}
\newcommand{\cP}{\mathcal{P}}
\newcommand{\cQ}{\mathcal{Q}}
\newcommand{\cR}{\mathcal{R}}
\newcommand{\cS}{\mathcal{S}}
\newcommand{\cT}{\mathcal{T}}
\newcommand{\cU}{\mathcal{U}}
\newcommand{\cV}{\mathcal{V}}
\newcommand{\cW}{\mathcal{W}}
\newcommand{\cX}{\mathcal{X}}
\newcommand{\cY}{\mathcal{Y}}
\newcommand{\cZ}{\mathcal{Z}}

\newcommand{\fg}{\mathfrak{g}}
\newcommand{\fh}{\mathfrak{h}}
\newcommand{\fp}{\mathfrak{p}}
\newcommand{\fm}{\mathfrak{m}}
\newcommand{\fn}{\mathfrak{n}}
\newcommand{\fX}{\mathfrak{X}}
\newcommand{\fY}{\mathfrak{Y}}
\newcommand{\fZ}{\mathfrak{Z}}

\newcommand{\bS}{\mathbf{S}}

% Header
\title{CO759 Project: All the Same, I Saw it First}
\author{Raymond Cheng, Sam Eisenstat}
\date{\today}
\begin{document}
\maketitle

\section{Teeth}
\subsection{Combs}
Let us quickly recall the setup. Let $G = (V,E)$ be a weighted graph. A
\textbf{comb} in $G$ consists of a \textbf{handle} $H \subsetneq V$ and $2k+1$,
$k \geq 1$, \textbf{teeth} $T_1,\ldots,T_{2k+1} \subset V$ such that $\abs{H}
\geq 3$, the $T_i$ are pairwise disjoint, $T_i \cap H \neq \varnothing$ and
$T_i \setminus H \neq \varnothing$. We shall also call a tooth $T_i$
\textbf{simple} if $\abs{T_i} = 2$.

The comb inequalities discovered by Chv\'atal and Gr\"otschel \& Padberg is
equivalent to the following: if $x$ is the characteristic vector of any tour in
$G$, then
\begin{equation}\label{ineq:comb}
  x(\delta(H)) + \sum^{2k+1}_{i = 1}x(\delta(T_i)) \geq 3(2k+1) + 1.
\end{equation}
Let us carefully examine~\eqref{ineq:comb} to see how we might find a violating comb.
Without loss, we may assume that the graph under consideration, although
fractional, satisfies the degree LP and all subtour inequalities. Moreover, by
going through the derivation of~\eqref{ineq:comb}, it also turns out that the if $x$
violates~\eqref{ineq:comb}, then the right hand side is smaller than the left hand side by
no more than $1$.

\subsection{Violating Combs}
To find a violating comb in $G$, we require a comb which makes the left hand
side of~\eqref{ineq:comb} as large as possible, while keeping the right side small. The
right hand side is determined completely by the number of teeth in the comb, so
whenever we construct a comb, we would like to have as many teeth as possible.

Suppose now that we have set $H \subset V$ which is to be the handle of some
comb.  The preceding discussion suggests that we should grow as many teeth out
of $H$ as possible. Let $v \in H$ be a vertex incident with edges
$e_1,\ldots,e_m$ such that the other end $u_i$ of $e_i$ is not in $H$. Consider
a tooth formed by $v$ along with some subset of the $u_i$, say $i =
1,\ldots,l$. This tooth will contribute
\begin{equation*}
  \sum^m_{i = 1}w(e_i) + \sum^l_{i = 1} (4 - 2w(e_i)) = 4l - \sum^l_{i = 1} w(e_i) + \sum^m_{i = l + 1}w(e_i) \geq 3l
\end{equation*}
on the left hand side of~\eqref{ineq:comb}, and a factor of $3$ on the right. The final
inequality holds with equality if and only if $w(e_i) = 1$ for $i = 1,\ldots,l$
and $l = m$. By the degree constraints, we also see that the equality conditions
hold only when $l = 1$.

\subsection{Best Teeth}
This analysis suggests a way to build a comb given a handle $H$. The best
possible teeth on $H$ are simple teeth in which the associated edge has weight
$1$. Suppose that $r$ edges $e_1,\ldots,e_r$ in $\delta(H)$ have edge weight
$1$.  The above analysis shows that in order to have any hope of finding a
violated comb inequality, we must add to our comb each simple tooth defined by
$e_i$. In the fortunate case that $n$ is odd and the $e_i$ are all the edges
incident with $H$, the resulting comb $C$ is a violating comb:
\begin{equation*}
  x(\delta(H)) + \sum^r_{i = 1}x(\delta(e_i)) = r + (4 - 2)r = 3r < 3r + 1.
\end{equation*}

\subsection{Lower Weight Teeth}
In general, either $r$ is even there are edges $e'_1,\ldots,e'_s$ that are of
weight strictly less than $1$. In the first case, nothing can be done and $H$
is not the handle defining a violating comb. In the second case, a judicious
choice of which $e_i'$ to include as a tooth in $C$ could still give a violated
comb. To get a feel for which teeth should be added, consider what~\eqref{ineq:comb} looks
like in this case so far. With the comb $C$ as above,
\begin{equation*}
  x(\delta(H)) + \sum^r_{i = 1}x(\delta(e_i)) = 3r + \sum^s_{i = 1}w(e_i') \geq 3r + 1.
\end{equation*}
Note that $\sum^s_{i = 1}w(e_i) \geq 1$ due to the degree constraints---in
fact, this sum is equal to the number $l$ of vertices $v_1,\ldots,v_l \in H$
incident with at least one of the edges $e_i'$. In fact, this tells us that we
must include one tooth for each vertex $v_i$. Consequently, in our
consideration, the number of vertices in $H$ which are incident to something
not in $H$ must be odd.

Consider the case in which we include exactly one edge incident with $v_i$ for
each $i$. By reindexing the edges, assume that the new teeth included are those
determined by $e_1',\ldots,e_l'$. The inequality~\eqref{ineq:comb} for this new comb $C'$
is then
\begin{equation*}
  3r + \sum^r_{i = 1} w(e_i') + \sum^l_{i = 1}(4 - 2w(e_j')) = 3(r+l) + \sum^r_{i = l+1}w(e_i') + \left(l - \sum^l_{i = 1}w(e_i')\right) \stackrel{?}{\geq} 3(r+l) + 1.
\end{equation*}
Then $C'$ is a violating comb if and only if $s + l$ is odd and
\begin{equation}\label{ineq:2}
  \sum^r_{i = l+1} w(e_i') + \left(l - \sum^l_{i = 1}w(e_i')\right) < 1.
\end{equation}
This inequality tells us that the edges $e_i'$ to be included should be chosen
with $w(e_i')$ maximal amongst those edges incident to the same vertex $v_j$.

\section{Handles}
In the above discussion, we assumed we were given a handle $H$ and then we
considered the teeth that must be taken to potentially construct a violated
comb.  But we now face the problem of finding the initial handle $H$. Our
simple heuristic to find $H$ will be guided by the properties of the edges
leaving $H$.

\subsection{Ignoring High and Low Weight Edges}
We saw that, ideally, the teeth of a comb are formed by edges of weight almost
$1$. Concretely, this means that $H$ should be an induced subgraph of $G$ such
that the weight of the edges joining $H$ with the vertices of $G$ should be
close to $1$. Such a subgraph can be found by considering the graph $G'$ whose
vertices are the same as those of $G$ and whose edges are those of $E$ with
weight smaller than some parameter $\beta$ close to $1$. The connected
components of $G'$ will then be candidates for handles.

But not all edges leaving a handle must be high weight edges.
Indeed~\eqref{ineq:2} says that handles may have low weight edges leaving it
too. Thus it would perhaps be more effective to choose another parameter
$\alpha$ close to $0$ and set $G'$ to be the subgraph of $G$ with the same
vertex set and edge set consisting of edges $e \in E$ satisfying $\alpha < w(e)
< \beta$. Denote by $G(\alpha,\beta) \coloneqq G'$ the subgraph formed in this
way. Connected components of $G(\alpha,\beta)$ are candidate handles.

\subsection{Oddness Condition}
Clearly, not all connected components of $G(\alpha,\beta)$ will be the handle
of a comb. For instance, connected components of size $1$ and $2$ cannot be a
handle. Also, we have seen that the handle of a violating comb must have have
an odd number of vertices which are incident to a edge leaving the handle.
Thus, amongst all connected components of $G(\alpha,\beta)$, those which can be
considered as potential handles are those components of size at least $3$ and
which have odd number of vertices on its border.
\end{document}
